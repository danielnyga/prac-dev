\documentclass[oribibl]{llncs}
%
\usepackage{makeidx}  % allows for indexgeneration

%%%%%CUSTOM PACKAGES%%%%%
\usepackage{graphicx}
\usepackage{float}
\usepackage{color}
\usepackage[usenames,dvipsnames]{xcolor}
\usepackage{framed}
\usepackage[utf8]{inputenc}
\usepackage{wrapfig}
\usepackage{alltt}
\renewcommand{\ttdefault}{txtt}

%%%FOR LATEXDRAW %%%%
 \usepackage[usenames,dvipsnames]{pstricks}
 \usepackage{epsfig}
 \usepackage{pst-grad} % For gradients
 \usepackage{pst-plot} % For axes
\usepackage{pifont} 


\newcommand{\comment}[1]{{\color{red} #1}}
\newcommand{\tick}{{\color{red}\ding{52}}} 
%%%%%%
%
\begin{document}
\begin{center}
  
{\huge \textbf{Paper review}}\\
\end{center}

This document wants to comprise all reviews of the ruleML paper submission, together
with the authors comments to them. This is to keep track of what has to be done in
the review phase, and also to prepare to possible questions during the conference.

\begin{alltt}
----------------------- REVIEW 1 ---------------------
PAPER: 31
TITLE: Controlled Natural Languages in Artificial Cognition: 
Language Generation and Action Formalization
AUTHORS: Nicholas Hubert Kirk, Daniel Nyga and Michael Beetz

OVERALL EVALUATION: -2 (reject)
REVIEWER'S CONFIDENCE: 4 (high)
Originality: 3 (fair)
Relevance to RuleML 2013: 3 (fair)
Significance: 3 (fair)
Technical Quality: 2 (poor)
Readability and Presentation: 2 (poor)

----------- REVIEW -----------
This paper investigates the use of a controlled natural language (CNL) 
as an interface language to a cognitive robot. The CNL is used as a 
specification language for actions and as a reporting language to 
support the resolution of ambiguities by a human. This is an interesting
 topic but unfortunately the paper is difficult to read since it is badly
  structured, lacks clarity and does not provide enough technical details.

I would start the paper with a presentation of the system architecture, 
then discuss how actions can be formalised so that they can be used by
 PRAC models and processed by MLN, and finally focus on the disambiguation 
 problem.

I doubt that the representation of the sentence "Flip the pancake" in 
Figure 2 is correct:
\comment{the Nyga et al. reference was not reviewed, perhaps? stress the 
importance of the PRAC system?}

   action_core(Flipping)
   action_role(p, Theme)
   isa(p, pancake)

Here the atom "action_core/1" is not connected in any way with the other 
two atoms. I think the authors should explain in more detail what's going 
on here.
\comment{the above relationships are not closed syntactic relationships. 
We should point better to the Nyga et al. reference?}
I don't understand why the authors use the Stanford parser to process the
 following sentence:

   The Instrument flips the THEME from FIXEDLOCATION.

Isn't it possible to process this sentence with the Attempto Controlled 
English parser?

\comment{The Stanford SYN Parser is already used by other parts of the PRAC
 process and has a finer granularity of syntactic types.}

Why is is necessary to generate the following two existential statements:

   There is a spatula.
   There is a tong.
\comment{it is our choice to define such as a preliminary doubt expression
 phase. We had to do our best given ACE shortcomings in verbalization.}

before the question for disambiguation is generated? Wouldn't it be better
 to directly ask the following question:

   With a spatula or a tong?
\comment{This sentence is not ACE, therefore it cannot be generated by
 the API we are using.}
\comment{ref. chapter 7: "In fact, the DRS construction of the
disambiguation cases had to account for the ACE construction rules (that can
present expressiveness limitations)".}

The answer to this question gives you the instrument you are looking for. 
However, this would require the following "controlled template":

   "The Actor flips the Theme with an Instrument."


----------------------- REVIEW 2 ---------------------
PAPER: 31
TITLE: Controlled Natural Languages in Artificial Cognition: Language 
Generation and Action Formalization
AUTHORS: Nicholas Hubert Kirk, Daniel Nyga and Michael Beetz

OVERALL EVALUATION: 2 (accept)
REVIEWER'S CONFIDENCE: 3 (medium)
Originality: 3 (fair)
Relevance to RuleML 2013: 4 (good)
Significance: 3 (fair)
Technical Quality: 4 (good)
Readability and Presentation: 4 (good)

----------- REVIEW -----------
The author(s) propose a novel use of CNL as a means for support 
human-computer interaction. the focus is on the formalization of 
actions trough the exchange of questions and clarification between 
human and robot in order to disambiguate and clarify the planning of 
tasks. the syntactical analysis and formulation grounds the construction
 of linguistic templates and the action formalization.
the paper is well written and clearly understandable. i'm not a deep 
expert of the field, but to my knowledge, the  application  of controlled 
languages in fields other than semantic web and multilingual searching, 
is a novelty and it seems to be quite interesting and promising.


----------------------- REVIEW 3 ---------------------
PAPER: 31
TITLE: Controlled Natural Languages in Artificial Cognition: Language 
Generation and Action Formalization
AUTHORS: Nicholas Hubert Kirk, Daniel Nyga and Michael Beetz

OVERALL EVALUATION: 2 (accept)
REVIEWER'S CONFIDENCE: 3 (medium)
Originality: 3 (fair)
Relevance to RuleML 2013: 3 (fair)
Significance: 4 (good)
Technical Quality: 4 (good)
Readability and Presentation: 5 (excellent)

----------- REVIEW -----------
Interesting and (as far as I know) original work, good to discuss at a 
workshop. The paper lacks technical details, so difficult to judge technical 
quality. The paper claims to present a resolution method to disambiguate 
action-specific information. It does present a short description of types 
of disambiguations and some examples of how a robot using ACE could try to 
solve it by asking for explicit clarification.

N.B.: missing reference in section 4.
\comment{solved.}

----------------------- REVIEW 4 ---------------------
PAPER: 31
TITLE: Controlled Natural Languages in Artificial Cognition: Language 
Generation and Action Formalization
AUTHORS: Nicholas Hubert Kirk, Daniel Nyga and Michael Beetz

OVERALL EVALUATION: 0 (borderline paper)
REVIEWER'S CONFIDENCE: 4 (high)
Originality: 3 (fair)
Relevance to RuleML 2013: 4 (good)
Significance: 3 (fair)
Technical Quality: 3 (fair)
Readability and Presentation: 3 (fair)

----------- REVIEW -----------
The paper claims to provide the means to "fill in" missing information 
using a sort of grammatical/semantic abduction.  This seems like a good 
idea.  However, the presentation does not do this idea a good service as 
it is unclear, laced with grammatical problems, not elaborated sufficiently 
where it needs to be, and does not take into consideration prior very closely
 related work.  It seems like very preliminary work.  Further particular comments.
 
 \comment{We should stress that we are mainly in the robotics context,
 and that current lexicon databases such as wordnet, etc., are not formal
 enough for our needs.}

p. 1 Was the LNCS format followed using the style file?  All the pages are 
numbered with roman numerals?
p. 1 "While…means," is not a clear phrase.
p. 1 requires to infer -> requires them to infer \tick
p. 1 in doing so -> in making executable robot plans
p. 2 "unadoperable" is not an English word \tick
p. 2  "inference might incur in limitations" is not a clear phrase. 
 What is meant?
p. 2  what instrument to be used -> what instrument is to be used \tick
p. 2 lacks of -> absent \tick
p. 3 An "action role" is what?  Can any action role combine with any 
action verb?  
The roles are clarified a bit later on, but they should be clarified 
where they 
are introduced.
\comment{TODO}
p. 3 In the definition of 'Flipping' and in general about thematic roles,
 the statements are inaccurate, overgeneral, and without reference.  
 There is research on building NL knowledge bases to 'fill in' information
  (VerbNet, FrameNet, PropNet) and also to address presuppositions 
  (see C&C/Boxer, for example).  Also, there is a great deal of work on 
  Textual Entailment on this subject.  This reflects a lack 
 of proper background research.  Start with Dowty "Word Meaning" and/or 
 search for research in linguistics on thematic roles.
p. 4 "Being this a" -> Being a \tick
p. 4 The DRS representation misses the thematic roles (which I thought 
were important). So you are saying that is why you need PARC?  
But C&C/Boxer has DRSs with thematic roles, so…the need for your 
tool is diminished?
p. 4 As you are already in a 'related work' section (3), what is the 
point of this section heading?
p. 4 ACE can work with some questions.  Not clear yet how 'doubt' is
 handled or identified. Clarified a bit later. 
p. 5 ACE proven to be used for sentence generation?  Awkward sentence
 here.  Do not know how useful ACE would be for generation.  The tool 
 in the Protege plug in is highly restricted in functionality.
p. 5 'case identification' Is this for disambiguation?
p. 5 With respect to "assignment is missing…."  Careful with your
 English grammar as this becomes wrong.  Needs to be: ..the role 
 assignment is missing, it is impossible to define 
a likely candidate or to identify…, or the optimal candidate is not
 available in context.
p. 5 About DRS templates, the idea seems to be a sort of abduction to the best information?
  The use of the DRS templates is not entirely clear to me.
p. 6 About "ACE sentence…configuration.", the proposal would require a significant 
augmentation to ACE, which does not have this capacity at the moment.
p. 6 From "The relationships…queried."  I do not understand this passage.
p. 7 The example for "flip" is very much like what VerbNet, FrameNet, and PropNet are 
intended to do.  What is the progress here?
p. 7 About "practical implementations", there is no indication from the paper that things 
have moved beyond the initial sketch of a design to a practical implementation.  If there 
is, then the paper needs to be much clearer about this point.
p. 7 "is a facilitated mean of knowledge" is ungrammatical. \tick
p. 7 "Infact" -> In fact \tick
p. 7 "Being the system purely a" -> The system being purely a \tick
p. 7 About "system outputs", I am not convinced about what the system outputs. \tick


----------------------- REVIEW 5 ---------------------
PAPER: 31
TITLE: Controlled Natural Languages in Artificial Cognition: Language Generation and 
Action Formalization
AUTHORS: Nicholas Hubert Kirk, Daniel Nyga and Michael Beetz

OVERALL EVALUATION: 1 (weak accept)
REVIEWER'S CONFIDENCE: 4 (high)
Originality: 4 (good)
Relevance to RuleML 2013: 3 (fair)
Significance: 2 (poor)
Technical Quality: 2 (poor)
Readability and Presentation: 2 (poor)

----------- REVIEW -----------
The idea is to infer missing information of a receipe from a classical
prototype of actions, their roles and the classes of the fillers, augmented
with subclass, part-of and probability information.

The architectural schema is described, involving CNL sentence parsing and
transformation into DRS, the comparision of DRS with the prototype,
generation of questions and integration of the answer. This schema is
clear and supports te plan of the paper.

A single example is used (a robot cooking a pankake), and I cannot determine
if the work is in progress or to be started. It also seems that the input
receipe must use Attempto, which conflicts with the previous affirmation that
it can be found on the web. The result is that we are left with no idea of the
feasability or the improvement of the robot behaviour.

p3:" according to the above equation"  which one ?
p7: what are the MLN templates ?
Infact --> In fact

\end{alltt}
\end{document}
